

% Русская локализация и кодировки
\usepackage{cmap}                % улучшенный поиск в PDF
\usepackage[T2A]{fontenc}        % кодировка шрифтов
\usepackage[utf8]{inputenc}      % кодировка исходного текста
\usepackage[russian]{babel}      % локализация и переносы

% Геометрия страницы: урезанные поля для максимальной печатной области
\usepackage[left=3cm,right=1.5cm,top=2cm,bottom=2cm,bindingoffset=0cm]{geometry}

% Интервалы и отступы
\usepackage{setspace}
\onehalfspacing                 % межстрочный интервал 1.5
\setlength{\parindent}{1.25cm}    % красная строка
\setlength{\parskip}{0pt}      % без дополнительного отступа между абзацами


% Работа с математикой
\usepackage{amsmath,amsfonts,amssymb,amsthm,mathtools}
\usepackage{icomma}

% Шрифты для математики
\usepackage{euscript}           % шрифт "Евклид"
\usepackage{mathrsfs}           % красивый мат. шрифт

% Русские списки
\usepackage{enumitem}
\makeatletter
\AddEnumerateCounter{\asbuk}{\russian@alph}{щ}
\makeatother

% Работа с картинками и таблицами
\usepackage{graphicx,caption,wrapfig,float}
\captionsetup{justification=centering, font=small}
\graphicspath{{images/}{images2/}}     % папки с картинками
\setlength{\fboxsep}{3pt}
\setlength{\fboxrule}{1pt}
\usepackage{array,tabularx,tabulary,booktabs,longtable,multirow}

% Код и листинги
\usepackage{listings}
\lstset{basicstyle=\small\ttfamily, columns=flexible, breaklines=true}

% TikZ
\usepackage{tikz}
\usetikzlibrary{graphs,graphs.standard}

% Гиперссылки
\usepackage[unicode, pdftex, hidelinks]{hyperref}

% Отступ перед первым абзацем в каждом разделе
\usepackage{indentfirst}

%% Верхний колонтитул
%\usepackage{fancyhdr}
%\pagestyle{fancy}
