\section{Результаты}
\label{sec:Chapter4} \index{Chapter4}

\subsection{Результаты численных расчетов}
Ниже представлены результаты моделирования для постановки с кольцевым источником с частотой 38 кГц. На рисунке \ref{fig:source} представлен типичный вид кольцевого излучателя на расчетной сетке.
\begin{figure}[H]
	\centering
	\includegraphics[width=0.8\textwidth]{source_bottom.png}
	\caption{Кольцевой излучатель, вид снизу на расчетную область}
	\label{fig:source}
\end{figure}

На рисунках \ref{fig:cos005}, \ref{fig:cos01}, \ref{fig:cos02}, \ref{fig:cos02} отображены результаты моделирования аберратора в виде полупериода синуса. Как можно видеть, при малой высоте профиля форма потенциала почти не отличается от равномерной, как на рис. \ref{fig:uniform}. При увеличении высоты в некоторый момент по бокам от вершины возникают две отдельные области минимума потенциала, вместо одной протяженной. 

\begin{figure}[H]
	\centering
	\includegraphics[width=0.8\textwidth]{cos005.png}
	\caption{Результат расчета для кольцевого пьезоэлемента 38 кГц, аберратор синусоидальной формы}
	\label{fig:cos005}
\end{figure}

\begin{figure}[H]
	\centering
	\includegraphics[width=0.8\textwidth]{cos01.png}
	\caption{Результат расчета для кольцевого пьезоэлемента 38 кГц, аберратор синусоидальной формы}
	\label{fig:cos01}
\end{figure} 

\begin{figure}[H]
	\centering
	\includegraphics[width=0.8\textwidth]{cos02.png}
	\caption{Результат расчета для кольцевого пьезоэлемента 38 кГц, аберратор синусоидальной формы}
	\label{fig:cos02}
\end{figure} 

\begin{figure}[H]
	\centering
	\includegraphics[width=0.8\textwidth]{cos03.png}
	\caption{Результат расчета для кольцевого пьезоэлемента 38 кГц, аберратор синусоидальной формы}
	\label{fig:cos03}
\end{figure}

\begin{figure}[H]
	\centering
	\includegraphics[width=0.8\textwidth]{uniform.png}
	\caption{Результат расчета для кольцевого пьезоэлемента 38 кГц без аберратора}
	\label{fig:uniform}
\end{figure}

Ниже на рисунках \ref{fig:par005}, \ref{fig:par001}, \ref{fig:par0005} приведены результаты моделирования параболического аберратора. Его ширина соответствует диаметру излучателя. Смотря на результаты, можно сделать вывод, что параболических аберратор сам по себе не сильно меняет картину потенциала.

\begin{figure}[H]
	\centering
	\includegraphics[width=0.8\textwidth]{par005.png}
	\caption{Результат расчета для кольцевого пьезоэлемента 38 кГц, аберратор параболической формы}
	\label{fig:par005}
\end{figure} 

\begin{figure}[H]
	\centering
	\includegraphics[width=0.8\textwidth]{par001.png}
	\caption{Результат расчета для кольцевого пьезоэлемента 38 кГц, аберратор параболической формы}
	\label{fig:par001}
\end{figure} 

\begin{figure}[H]
	\centering
	\includegraphics[width=0.8\textwidth]{par0005.png}
	\caption{Результат расчета для кольцевого пьезоэлемента 38 кГц, аберратор параболической формы}
	\label{fig:par0005}
\end{figure}

Наконец, для цилиндрического пьезоэлемента на 108 кГц рассмотрен аберратор в виде синусоидального профиля. Результаты представлены на рисунках \ref{fig:sin1}, \ref{fig:sin2}, \ref{fig:sin3}, \ref{fig:sin4}. Как видно из рисунков, количество периодов слабо влияет на получаемую картину потенциала. Интересным кажется случай на рис. \ref{fig:sin2} --- в некоторых областях форма потенциала похожа на обратную форму аберратора.

\begin{figure}[H]
	\centering
	\includegraphics[width=0.8\textwidth]{sin1.png}
	\caption{Результат расчета для цилиндрического пьезоэлемента 108 кГц, аберратор синусоидальной формы}
	\label{fig:sin1}
\end{figure} 

\begin{figure}[H]
	\centering
	\includegraphics[width=0.8\textwidth]{sin2.png}
	\caption{Результат расчета для цилиндрического пьезоэлемента 108 кГц, аберратор синусоидальной формы}
	\label{fig:sin2}
\end{figure} 

\begin{figure}[H]
	\centering
	\includegraphics[width=0.8\textwidth]{sin3.png}
	\caption{Результат расчета для цилиндрического пьезоэлемента 108 кГц, аберратор синусоидальной формы}
	\label{fig:sin3}
\end{figure}

\begin{figure}[H]
	\centering
	\includegraphics[width=0.8\textwidth]{sin4.png}
	\caption{Результат расчета для цилиндрического пьезоэлемента 108 кГц, аберратор синусоидальной формы}
	\label{fig:sin4}
\end{figure}

\subsection{Результаты физического эксперимента}
К сожалению, автору не удалось получить устойчивую картину акустической левитации. Ни один из трех используемых пьезоэлементов не создал достаточно мощное поле. Наглядным показателем является рябь на воде при включении, однако она не наблюдалась. Это может быть связано с различными факторами. Например, неточно сконструированный согласующий слой, а именно наличие пузырьков в толще эпоксидной смолы или погрешность толщины. Однако наиболее вероятной причиной является недостаточная мощность усилителя. Об этом свидетельствует тот факт, что при работе он быстро (около 30 секунд) перегревается, ток после старта начинает падать, а синусоидальный сигнал искажается. При этом в слышимом диапазоне все три излучателя издавали достаточно громкий звук при работе, но, судя по всему, это не является показателем.

Самый близкий к удачному результат был получен с кольцевым пьезоэлементом на частоте 60 кГц при работе в сильно соленой воде. Была предпринята попытка довести плотность воды почти до плотности PLA-пластика, чтобы частички обладали почти нейтральной плавучестью. При опускании с поверхности, на несколько секунд наблюдалась практически полная остановка одной из частиц на некотором расстоянии над излучателем, но вскоре после этого частица утонула. На рисунке \ref{fig:final} можно наблюдать этот момент. Соленая вода была мутной, поэтому качество фотографии оставляет желать лучшего.

\begin{figure}[H]
	\centering
	\includegraphics[width=1\textwidth]{final.jpg}
	\caption{Момент замедления частицы над излучателем}
	\label{fig:final}
\end{figure} 

В связи с тем, что не удалось получить устойчивую картину в простейшем случае, физических экспериментов с аберратором проведено не было. 
\newpage
