\section{Заключение}
\label{sec:Chapter5} \index{Chapter5}
Подводя итоги, перечислим основные результаты работы:
\begin{enumerate}
	\item В ходе численного моделирования были получены различные картины потенциала Горькова для разных форм аберратора: полупериод косинуса, парабола, периоды синуса. В общем, можно сказать, что управление акустическим полем с помощью аберраторов возможно, но с отражателем плоской формы не очень эффективно.
	\item Хотя физический эксперимент провести не удалось, получены ценные знания и опыт по созданию установки для акустической левитации. Было выявлено, что схема на основе усилителя мощностью 25 Вт для звуковой аппаратуры не достаточно для получения устойчивой картины стоячих волн в воде в условиях, рассмотренных в работе. Для хорошего усиления сигнала для пьезоэлемента нужны специализированные усилители для ультразвуковых частот. Помимо этого автором получен опыт пайки пьезоэлементов, печати форм и заливки их эпоксидной смолой. Было подтверждено, что эпоксидная смола хорошо подходит для электроизоляции и герметизации от воды. 
\end{enumerate}
\newpage
