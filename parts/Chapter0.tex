\section{Введение}
\label{sec:Chapter0} \index{Chapter0}
\subsection{Актуальность}
Акустическая левитация представляет собой бесконтактный метод удержания и манипуляции телами в жидкой или газообразной среде за счёт действия сил акустического давления, возникающих при взаимодействии акустических волн с объектом. За последние десятилетия эта технология приобрела широкое признание в различных областях науки и техники благодаря своей способности минимизировать механические и химические взаимодействия с удерживаемым образцом. Она находит применение в многих инженерных приложениях, например, в фармацевтической области \cite{appliance_medicine}, машиностроении \cite{appliance_bearings} и прецизионной робототехнике \cite{appliance_robot}. В частности, одним из приложений является необходимость бесконтактного позиционирования биологических тканей в задаче биопечати. 

Существуют различные технические способы управления акустическим полем. К примеру, использование отражателей сложной формы в совокупности с массивом излучателей исследовано в \cite{nature_acoustics}. Другим подходом оптимизации акустических левитаторов является использование акустических аберраторов — специальных элементов, способных искажать фазу или амплитуду акустических волн для формирования заданного пространственного распределения давления. Такие аберраторы позволяют целенаправленно управлять звуковым полем, создавать нужную форму потенциала и задавать положение левитируемых объектов, не прибегая к использованию сложных массивов излучателей.

\subsection{Цель работы}

Настоящая работа посвящена численному моделированию акустической левитации с учётом применения аберрирующих элементов. Исследуется влияние параметров аберраторов на структуру акустического поля, полученная картина потенциала Горького и реальная устойчивость положения левитируемых объектов. Результаты могут быть использованы при проектировании перспективных систем для микро- и макроманипуляции. Для достижения поставленной цели были поставлены следующие задачи:
\begin{enumerate}
	\item Разработка и создание экспериментальной установки, позволяющей проводить физические эксперименты по акустический левитации в жидкости с возможностью использования различных излучателей
	\item Разработка или выбор готового программного комплекса, с помощью которого проводится моделирование физического эксперимента
\end{enumerate}

\newpage
