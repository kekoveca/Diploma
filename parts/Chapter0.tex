\section{Введение}
\label{sec:Chapter0} \index{Chapter0}

Акустическая левитация представляет собой бесконтактный метод удержания и манипуляции телами в жидкой или газообразной среде за счёт действия сил акустического давления, возникающих при взаимодействии акустических волн с объектом. За последние десятилетия эта технология приобрела широкое признание в различных областях науки и техники благодаря своей способности минимизировать механические и химические взаимодействия с удерживаемым образцом. Она находит применение в многих инженерных приложениях, например, в фармацевтической области \cite{appliance_medicine}, машиностроении \cite{appliance_bearings} и прецизионной робототехнике \cite{appliance_robot}.

В частности, акустическая левитация активно используется для бесконтактной кристаллизации, управления реакциями в микрокаплях, транспортировки чувствительных материалов и изучения фазовых переходов в условиях изоляции от твёрдых поверхностей. Однако эффективность и устойчивость таких систем в значительной степени зависят от конфигурации акустического поля, его симметрии, фокусировки и формы пространственного распределения энергии.

Одним из перспективных направлений оптимизации акустических левитаторов является использование акустических аберраторов — специально сконструированных элементов, способных искажать фазу или амплитуду ультразвуковых волн для формирования заданного пространственного распределения акустического давления. Такие аберраторы позволяют целенаправленно модифицировать звуковое поле, создавать локальные экстремумы потенциала и управлять положением левитируемых объектов с повышенной точностью.

Настоящая работа посвящена численному моделированию акустической левитации с учётом применения аберрирующих элементов. Исследуется влияние параметров аберраторов на структуру акустического поля, распределение радиационных сил и устойчивость положения левитируемых объектов. Результаты могут быть использованы при проектировании адаптивных акустических ловушек и систем бесконтактной микро- и макроманипуляции.

\newpage
