\section{Введение}
\label{sec:Chapter0} \index{Chapter0}
\subsection{Актуальность}
Акустическая левитация представляет собой бесконтактный метод удержания и манипуляции телами в жидкой или газообразной среде за счёт действия сил акустического давления, возникающих при взаимодействии акустических волн с объектом. За последние десятилетия эта технология приобрела широкое признание в различных областях науки и техники благодаря своей способности минимизировать механические и химические взаимодействия с удерживаемым образцом. Она находит применение в многих инженерных приложениях, например, в фармацевтической области \cite{appliance_medicine}, машиностроении \cite{appliance_bearings}, прецизионной робототехнике \cite{appliance_robot}, \cite{damping}, анализе материалов \cite{materials}, а также биологии \cite{bio}. Примером численного моделирования акустической левитации в простейшем случае является работа \cite{acoustics_sim}.

В частности, одним из приложений является необходимость бесконтактного позиционирования биологических тканей в задаче биопечати. Например, задача биопечати коленного хряща актуальна в связи с ограниченной способностью хрящевой ткани к регенерации и высокой распространенностью суставных повреждений. Современные методы лечения, пример которого представлен на рис. \ref{fig:protez} \cite{protez}, имеют ряд недостатков, таких как отторжение, износ, плохую интеграцию. Рентген-снимки коленного сустава после протезирования представлены на рис. \ref{fig:knee2} и \ref{fig:knee3}. Иллюстрации взяты из работ \cite{knee2} и \cite{knee3} соответственно. Технологии 3D-биопечати позволяют создавать более совершенные персонализированные хрящевые конструкции, что делает их перспективными для замещения и восстановления хряща коленного сустава. Форма хряща обуславливает форму моделируемого в настоящей работе аберратора.

\begin{figure}[H]
	\centering
	\includegraphics[width=1\textwidth]{knee.jpg}
	\caption{Коленный сустав, иллюстрация взята из книги \cite{knee1}}
	\label{fig:knee}
\end{figure}

\begin{figure}[H]
	\centering
	\includegraphics[width=0.4\textwidth]{protez.jpg}
	\caption{Пример протеза для коленного сустава}
	\label{fig:protez}
\end{figure} 

\begin{figure}[H]
	\centering
	\includegraphics[width=0.5\textwidth]{knee2.jpg}
	\caption{Рентген-снимок протеза для коленного сустава}
	\label{fig:knee2}
\end{figure} 

\begin{figure}[H]
	\centering
	\includegraphics[width=0.5\textwidth]{knee3.jpg}
	\caption{Рентген-снимок протеза для коленного сустава}
	\label{fig:knee3}
\end{figure} 

Существуют различные технические способы управления акустическим полем. К примеру, использование отражателей сложной формы в совокупности с массивом излучателей исследовано в \cite{nature_acoustics}. Другим подходом оптимизации акустических левитаторов является использование акустических аберраторов — специальных элементов, способных искажать фазу или амплитуду акустических волн для формирования заданного пространственного распределения давления. Такие аберраторы позволяют целенаправленно управлять звуковым полем, создавать нужную форму потенциала и задавать положение левитируемых объектов, не прибегая к использованию сложных массивов излучателей.

\subsection{Цель работы}

Настоящая работа посвящена численному моделированию акустической левитации с учётом применения аберрирующих элементов. Исследуется влияние параметров аберраторов на структуру акустического поля, полученная картина потенциала Горькова и реальная устойчивость положения левитируемых объектов. Результаты могут быть использованы при проектировании перспективных систем для микро- и макроманипуляции. Для достижения поставленной цели были поставлены следующие задачи:
\begin{enumerate}
	\item Разработка и создание экспериментальной установки, позволяющей проводить физические эксперименты по акустический левитации в жидкости с возможностью использования различных излучателей
	\item Разработка или выбор готового программного комплекса, с помощью которого проводится моделирование физического эксперимента
\end{enumerate}

\newpage
