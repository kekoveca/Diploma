\section{Математическая модель}
\label{sec:Chapter1} \index{Chapter1}

\subsection{Система уравнений акустики}
Основой для описания исследуемой среды является модель линейной упругости. Тогда в общем случае, согласно \cite{kondaurov}

\begin{equation}
	\begin{aligned}
		\rho \dot{\vec{v}} & =\nabla \cdot \mathbb{T}+\vec{f} \\
		\dot{\mathbb{T}} & =\mathbf{q}: \dot{\varepsilon}+\mathbf{F}
	\end{aligned}
\end{equation}

Здесь $\rho$ - плотность в данной точке, $\vec{v}$ - скорость частиц в данной точке, $\mathbb{T}$ - тензор напряжений в данной точке, $\varepsilon$ - тензор деформаций в данной точке, $\vec{f}$ и $\mathbf{F}$ - силы, $\mathbf{q}$ - тензор упругих коэффициентов. Далее везде силы примем равными нулю, согласно постановке задачи. Пользуясь малостью смещения, используем выражения для тензора малых деформаций

\begin{equation}
	\varepsilon=\frac{1}{2}\left(\nabla \otimes \vec{r}+\vec{r} \otimes \nabla\right)
\end{equation}

Здесь $\vec{r}$ - вектор смещения в данной точке, $\otimes$ - оператор тензорного умножения. Дифференцируя это выражения по времени и подставляя в исходную систему, получаем

\begin{equation}
	\begin{aligned}
		\rho \dot{\vec{v}} & =\nabla \cdot \mathbb{T} \\
		\dot{\mathbb{T}} & =\frac{1}{2} \mathbf{q}:\left(\nabla \otimes \vec{v}+ \vec{v} \otimes \nabla \right)
	\end{aligned}
\end{equation}

Воспользуемся теперь тем, что исследуемая среда изотропна, то есть свойства одинаковы для любого направления в пространстве. Известно, что для такого материала количество независимых коэффициентов в тензоре $\mathbf{q}$ равно двум. Их называют параметры Ламе и обычно обозначают $\lambda$, $\mu$. Полезно также представить их выражения через модуль Юнга $E$ и коэффициент Пуассона $\nu$

\begin{equation}
	\begin{aligned}
		\lambda & =\frac{E \nu}{(1+\nu)(1-2 \nu)} \\
		\mu & =\frac{E}{2(1+\nu)} .
	\end{aligned}
\end{equation}

Итого, для изотропного случая можно записать

\begin{equation}
	\label{eq:isotropic}
	\begin{aligned}
		\rho \dot{\vec{v}} & =\nabla \cdot \mathbb{T} \\
		\dot{\mathbb{T}} & =\lambda(\nabla \cdot \vec{v}) \mathbb{I}+\mu\left(\nabla \otimes \vec{v}+ \vec{v} \otimes \nabla\right)
	\end{aligned}
\end{equation}

Остается последний шаг преобразования. Применим акустическое приближение: тензор напряжений зависит только от одного скалярного параметра $p$, который называют давлением, и выражается через единичный тензор $\mathbb{I}$. Модуль сдвига $\mu$ считается равен нулю.

\begin{equation}
	\begin{aligned}
		\mathbb{T} = -p \mathbb{I}
	\end{aligned}
\end{equation}

Эта модель хорошо подходит для описания жидкостей, что и требуется в настоящей работе. Традиционно переобозначим $\lambda$ за $K$ --- модуль всестороннего сжатия. Наконец,  подставляя это выражение в \ref{eq:isotropic}, получаем

\begin{equation}
	\begin{aligned}
		\rho \dot{\vec{v}} & =-\nabla (p \mathbb{I}) \\
		\dot{p} & =-K(\nabla \cdot \vec{v})
	\end{aligned}
\end{equation}

\subsection{Выражения для коэффициентов прохождения и отражения волны}

Для некоторых дальнейших рассуждений полезно понимать, какая часть волны проходит через контактную границу, а какая отражается обратно. В \cite{kazakov} приведено более подробное рассуждение, в настоящей же работе ограничимся соотношениями для нормально падающей на контактную границу волны:

\begin{equation}
	\label{eq:refl}
	\begin{aligned}
		R=\frac{Z_2-Z_1}{Z_2+Z_1} \\ 
		T=\frac{2 Z_2}{Z_2+Z_1}
	\end{aligned}
\end{equation}

Здесь $R$ - коэффициент отражения по амплитуде, $T$ - коэффициент прохождения по амплитуде. Возводя их в квадрат можно получить коэффициенты по энергии, но они не понадобятся. Индексам  "1"\ и "2"\ соответствуют величины в среде из которой и в которую бежит волна соответственно. Также важное обозначение 
\begin{equation}
	Z = \sqrt{\frac{K}{\rho}} = \rho c
	\label{impedance}
\end{equation}
--- акустический импеданс среды, величина, характеризующая сопротивление среды распространению звуковых волн, $c$ --- скорость звука в среде.

\subsection{Потенциал Горькова}
Потенциал Горькова --- это скалярная функция, отрицательный градиент которой описывает силу, действующую на малую частицу сферической формы в акустическом поле. Оригинальную формулу можно найти в \cite{gorkov}, а в настоящей работе будем использовать безразмерный потенциал, удобный для реализации в программном коде:

\begin{equation}
	U = \frac{\sqrt{ \langle p'^2 \rangle }}{3} - \frac{\sqrt{ \langle v'^2 \rangle }}{2}
\end{equation}

\begin{equation}
	\langle p'^2 \rangle = \langle p^2 \rangle - \langle p \rangle^2, \quad
	\langle v'^2 \rangle = \langle \mathbf{v}^2 \rangle - \langle \mathbf{v} \rangle^2
\end{equation}
Здесь давление и скорость приведены к безразмерной форме делением на $\rho c$ и $c$ соответственно.

\newpage
