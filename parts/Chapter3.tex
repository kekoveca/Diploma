\section{Постановка задачи и эксперимента}
\label{sec:Chapter3} \index{Chapter3}

\subsection{Постановка задачи}
Как было сказано ранее, основной целью работы является изучение применения аберраторов для акустической левитации в воде. Исходя из этого были сформулированы требования к экспериментальной установке:
\begin{itemize}
	\item Главным инструментом для контроля формы распределения акустического давления является аберратор, то есть не используются матрицы из излучателей (используется один излучатель), а также отражатели в форме, отличной от плоской.
	\item В качестве излучателя используется пьезоэлемент. В данном случае это решение безальтернативное.
	\item Диапазон доступных частот в 10--100 кГц и величина скорости звука в воде накладывают ограничения на размер резервуара для воды.
\end{itemize}
В качестве формы аберратора был выбран достаточно простой синусоидальный профиль, чтобы его можно было изготовить не прибегая к сложным технологиям. 
\subsection{Пьезоэлектрический эффект}
Как было сказано ранее, в экспериментальной установке излучателем является пьезоэлемент, так что следует привести краткие сведения о том, что это такое. Пьезоэлектрический эффект представляет собой физическое явление, заключающееся в возникновении электрического заряда на поверхности некоторых кристаллических материалов при их механическом деформировании. Этот эффект наблюдается в веществах с анизотропной кристаллической структурой, не обладающей центром симметрии. Он проявляется в двух формах: прямой и обратной.

Прямой пьезоэлектрический эффект заключается в генерации электрического потенциала в ответ на приложенное механическое воздействие (сжатие, растяжение, изгиб и др.). Это происходит вследствие изменения положения ионов в кристаллической решетке, что приводит к поляризации материала и возникновению поверхностного заряда. Сила возникшего электрического поля пропорциональна величине механической деформации. Обратный пьезоэлектрический эффект представляет собой деформацию кристалла под действием внешнего электрического поля. Это связано с перераспределением внутренних сил в кристаллической решётке, вызывающим изменение её геометрических размеров.

Наиболее ярко выраженные пьезоэлектрические свойства наблюдаются у таких материалов, как кварц, турмалин, титанат бария, цирконат-титанат свинца (PZT, пьезоэлементы из него используются в настоящей работе), а также у некоторых полимеров, например поливинилиденфторида (PVDF).

Благодаря способности преобразовывать электрический сигнал в механический такой же формы, пьезоэлемент отлично подходит для генерации синусоидальных волн, которые необходимы для акустической левитации.
\subsection{Создание экспериментальной установки}
Проектирование и создание экспериментальной установки оказалось нетривиальной задачей. Готовые решения не рассматривались, так как задача довольно специфическая, поэтому они дорого стоят или их тяжело найти и купить. Было принято создавать установку из условно подручных средств. Основными частями экспериментальной установки являются:
\begin{itemize}
	\item Резервуар с водой --- среда для левитации частиц
	\item Пьезоэлектрический элемент в качестве генератора волн в среде
	\item Источник синусоидального сигнала
	\item Усилитель для источника сигнала
	\item Источник питания для усилителя
	\item Аберратор 
	\item Левитирующие частицы
\end{itemize}
Расскажем по порядку про каждый из пунктов. В качестве резервуара был выбран пластмассовый контейнер размером приблизительно 20x18x15 см. Выбор обусловлен простотой приобретения и наличием прозрачных стенок для визуального наблюдения. Также размер позволяет уместить несколько длин волн в диапазоне 10--100 кГц, так как изначально не было ясно, какая частота в итоге будет использована. 

В роли пьезоэлементов были опробованы три образца кристаллов с разной резонансной частотой:
\begin{itemize}
	\item Цилиндр 14x12 мм, 108 кГц
	\item Кольцо 46x16x9 мм, 38 кГц
	\item Кольцо 22x10x8 мм, 60 кГц
\end{itemize}
К образцам было необходимо припаять провода и изолировать от воды во избежание короткого замыкания. Пайка производилась на минимально доступной температуре паяльной станции $270^\circ\mathrm{C}$, так как при перегреве кристаллы теряют свои пьезоэлектрические свойства. Для электрической изоляции использовалась эпоксидная смола. Здесь стоит сказать еще об одной важной особенности, а именно о согласующих слоях между излучателем и средой. Если у двух сред сильно отличается акустический импеданс Z, то большая часть волны может отражаться от границы, что сильно ухудшает картину потенциала Горькова (это наглядно можно видеть из формул \ref{eq:refl}). Некоторую информацию об этой теме можно найти, например, в \cite{matching_layers}. Суть состоит в том, что при правильном выборе материала и его толщины можно добиться снижения отражения от контактной границы. В настоящей работе для удобства согласующий слой сделан из эпоксидной смолы толщиной $\frac{v}{4f}$, где $v$ и $f$ --- скорость звука и частота в смоле. Из-за различия частоты для разных пьезоэлементов толщина получалась разная. Формы для отливки были напечатаны на 3D принтере из пластика PETG, так как автор посчитал это самым простым способом. 

\begin{figure}[H]
	\centering
	\includegraphics[width=0.8\textwidth]{piezo_cilinder.jpg}
	\caption{Цилиндрический пьезоэлемент, 108 кГц}
	\label{fig:cilindr}
\end{figure}

\begin{figure}[H]
	\centering
	\includegraphics[width=0.8\textwidth]{piezo_ring.jpg}
	\caption{Кольцевой пьезоэлемент, 38 кГц}
	\label{fig:ring}
\end{figure}

\begin{figure}[H]
	\centering
	\includegraphics[width=0.8\textwidth]{piezo_ring_small.jpg}
	\caption{Кольцевой пьезоэлемент, 60 кГц}
	\label{fig:ring_small}
\end{figure}

Для генерации синусоидального сигнала было решено использовать микроконтроллер. Такой выбор обусловлен тем, что автор знаком с программированием микроконтроллеров серии STM32, а остальные способы, например аналоговое преобразование ШИМ-сигнала требуют более глубоких знаний в области радиоэлектроники. Также синтез сигнала на микроконтроллере позволяет в реальном времени настраивать частоту, что очень удобно. Минусом данного подхода является тот факт, что у любого цифрового сигнала неограниченный спектр, и, вообще говоря, нужно ставить фильтр на выход генератора. В настоящей работе это сделано не было и влияние этого фактора не изучалось. Регулировка частоты реализована в виде отправки команды с компьютера на микроконтроллер.

Ввиду отсутствия доступных готовых решений, усилитель был создан в виде платы на основе микросхемы LM1875T. Вообще говоря, эта микросхема является усилителем для звуковой аппаратуры и не подходит ультразвуковых излучателей, однако годится для кратковременного использования на интересующих частотах и является доступной. Так как автор не является специалистом в области радиотехники, в роли схемы была взята соответствующая из руководства для LM1875T \cite{lm1875} (рис. \ref{fig:lm1875}). Фото получившейся платы представлено на рис. \ref{fig:ampl}.

\begin{figure}[H]
	\centering
	\includegraphics[width=0.8\textwidth]{lm1875.png}
	\caption{Электрическая схема усилителя мощности, используемого в настоящей работе}
	\label{fig:lm1875}
\end{figure}

\begin{figure}[H]
	\centering
	\includegraphics[width=0.9\textwidth]{ampl.jpg}
	\caption{Плата усилителя мощности}
	\label{fig:ampl}
\end{figure}

В качестве источника питания был взят простой лабораторный. Возможно, для минимизации помех в синусоидальном сигнале стоило взять более совершенный источник, однако этот вопрос не изучался в настоящей работе. 

Аберратор --- главная часть данной работы --- планировалось изготавливать либо путем отливки силикона в формах, напечатанных на 3D принтере, либо печатать сразу целиком.

В роли левитирующих частиц могли выступать икринки, манная каша или шарики из пластика PLA. Важно, чтобы частицы в спокойной воде тонули не очень быстро, не растворялись и не теряли сферическую форму.

На рисунке \ref{fig:experimental} представлено фото экспериментальной установки. Слева снизу --- сосуд с пьезоэлементом на дне, около него источник питания. Справа от источника --- осциллограф, который подключен к выходу усилителя и позволяет контролировать сигнал, приходящий на излучатель. Посередине находится плата усилителя, к которой подключен микроконтроллер (небольшая черная плата), источник питания и пьезоэлемент. Микроконтроллер соединен через адаптер с компьютером.

\begin{figure}[H]
	\centering
	\includegraphics[width=1\textwidth]{experimental.jpg}
	\caption{Фото экспериментальной установки}
	\label{fig:experimental}
\end{figure} 

\subsection{Постановка численного эксперимента}
В численном эксперименте интерес представляло посмотреть на потенциал Горькова для различных форм аберратора, различных частот и форм излучателя, в соответствии с созданной экспериментальной установкой. Параметры реологии были одинаковые для всех расчетов, здесь "п"\ --- пластик, материал аберратора, "в"\ --- вода:
\begin{table}[H]
	\noindent Таблица 1 – Акустические параметры для аберратора и воды
	
	\begin{center}
		\begin{tabular}{|c|c|c|}
			\hline
			Материал & $c$, м/с & $\rho$, кг/м$^3$ \\
			\hline
			Пластик & 2200 & 1270 \\
			Вода & 1500 & 1000 \\
			\hline
		\end{tabular}
	\end{center}
\end{table}
Шаг сетки был разным для разных серий численных опытов. Его выбор в каждом случае обусловлен балансом между скоростью расчета и разрешением геометрии. Сетка во всех случаях ортогональная. Для кольцевого пьезоэлемента с частотой 38 кГц в роли расчетной области была взята прямоугольная область реальным размером 10x10x10 см. Расчетная область соответствует центральной части сосуда. Граничные условия ставятся следующие:
\begin{itemize}
	\item На боковых стенках --- поглощение. Предполагается, что в реальном эксперименте стенки далеко и отражение от них мало.
	\item На верхней и нижней границе --- отражение. Это соответствует действительности, так как в реальности происходит отражение от пластикового дна снизу и свободной границы воды сверху.
\end{itemize}
Пьезоэлемент моделируется источником на нижней границе. Задается давление, зависящее от времени по гармоническому закону. Количество шагов по времени выбирается исходя из следующего условия: волна от источника должна дойти до верхней границы два раза. Это позволяет получить приемлемое усреднение для расчета потенциала Горькова, при этом экономится время выполнения программы. В качестве аберратора моделируются две формы параболическая и полупериод косинуса.

Для цилиндрического пьезоэлемента с частотой 108 кГц была выбрана расчетная область размером 5x5x5 см. Уменьшение области связано с тем, что для большей частоты образуется больше минимумов в потенциале Горькова. Формой для аберратора выступил синус с различным числом периодов. Остальные параметры такие же, как в постановке для кольцевого излучателя.



\newpage
