\section{Постановка задачи и эксперимента}
\label{sec:Chapter3} \index{Chapter3}

\subsection{Постановка задачи}
Как было сказано ранее, основной целью работы является изучение применения аберраторов для акустической левитации в воде. Исходя из этого были сформулированы требования к экспериментальной установке:
\begin{itemize}
	\item Главным инструментом для контроля формы распределения акустического давления является аберратор, то есть не используются матрицы из излучателей (используется один излучатель), а так же отражатели в форме, отличной от плоской.
	\item В качестве излучателя используется пьезоэлемент. В данном случае это решение безальтернативное.
	\item Диапазон доступных частот в 10--100 кГц и величина скорости звука в воде накладывают ограничения на размер резервуара для воды.
\end{itemize}
В качестве формы аберратора был выбран достаточно простой синусоидальный профиль, чтобы его можно было изготовить не прибегая к сложным технологиям. 
\subsection{Пьезоэлектрический эффект}
Как было сказано ранее, в экспериментальной установке излучателем является пьезоэлемент, так что следует привести краткие сведения о том, что это такое. Пьезоэлектрический эффект представляет собой физическое явление, заключающееся в возникновении электрического заряда на поверхности некоторых кристаллических материалов при их механическом деформировании. Этот эффект наблюдается в веществах с анизотропной кристаллической структурой, не обладающей центром симметрии. Он проявляется в двух формах: прямой и обратной.

Прямой пьезоэлектрический эффект заключается в генерации электрического потенциала в ответ на приложенное механическое воздействие (сжатие, растяжение, изгиб и др.). Это происходит вследствие изменения положения ионов в кристаллической решётке, что приводит к поляризации материала и возникновению поверхностного заряда. Сила возникшего электрического поля пропорциональна величине механической деформации. Обратный пьезоэлектрический эффект представляет собой деформацию кристалла под действием внешнего электрического поля. Это связано с перераспределением внутренних сил в кристаллической решётке, вызывающим изменение её геометрических размеров.

Наиболее ярко выраженные пьезоэлектрические свойства наблюдаются у таких материалов, как кварц, турмалин, титанат бария, цирконат-титанат свинца (PZT, пьезоэлементы из него используются в настоящей работе), а также у некоторых полимеров, например поливинилиденфторида (PVDF).

Благодаря способности преобразовывать электрический сигнал в механический такой же формы, пьезоэлемент отлично подходит для генерации синусоидальных волн, которые необходимы для акустической левитации.
\subsection{Создание экспериментальной установки}

\newpage
